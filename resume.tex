\documentclass[a4paper,10pt]{article}

% Useful Packages
\usepackage{marvosym, verbatim}
\usepackage{xunicode,xltxtra,url,parskip} 	% other packages for formatting
\RequirePackage{color}
\usepackage[big]{layaureo} 				% better formatting of the A4 page
\usepackage{titlesec}					% custom \section
\usepackage{fontspec}
\usepackage{setspace}

% Setup hyperref package, and colors for links
\usepackage{hyperref}
\definecolor{links}{rgb}{0,0.15,0.6}
\hypersetup{colorlinks,linkcolor=,urlcolor=links}

% Fonts
\defaultfontfeatures{Mapping=tex-text}
\setmainfont[
SmallCapsFont = Fontin-SmallCaps.otf,
BoldFont = Fontin-Bold.otf,
ItalicFont = Fontin-Italic.otf
]
{Fontin.otf}

% Tweak a bit the margin
\addtolength{\voffset}{-1.4cm}
\addtolength{\textheight}{2.8cm}
\addtolength{\hoffset}{-1.3cm}
\addtolength{\textwidth}{2.6cm}

% Custom titles
\titleformat{\section}{\vspace{-0.5mm}\Large\scshape\raggedright}{}{0em}{}[{\titlerule[.8pt]}]
\titleformat{\subsection}{\scshape\raggedright}{}{0em}{}[{\titlerule[.4pt]}]
\titlespacing{\section}{0pt}{0pt}{0pt}
\titlespacing{\subsection}{0pt}{2pt}{2pt}

%--------------------BEGIN DOCUMENT----------------------
\begin{document}
\pagestyle{empty} % non-numbered pages
\font\fb=''[cmr10]'' % for use with \LaTeX command
\setlength{\tabcolsep}{5pt}
\renewcommand{\arraystretch}{0}

\definecolor{light-gray}{gray}{0.1}
\color{light-gray}

%--------------------TITLE-------------------------------
\newcommand{\email}[0]{\href{mailto:brandcomin@gmail.com}{brandcomin@gmail.com}}
\newcommand{\github}[0]{\href{https://github.com/BrandonComins}{github.com/BrandonComins}}
\newcommand{\phone}[0]{(818) 564-9626}
\newcommand{\titlesep}[0]{ • }
\newcommand{\titlename}[2]{\par{\centering{\Huge #1 \textsc{#2}}\par}}
\newcommand{\info}[1]{\centering{\ensuremath{\overline{\underline{\mbox{#1}}}}}}
\titlename{Brandon}{Comins}
\info{\email\titlesep\phone\titlesep\github}


%--------------------CUSTOM OPERATORS--------------------
\newcommand{\term}[1]{\texttt{\uppercase{#1}}}

\newcommand{\resumeSec}[2]{\section{#1}\begin{tabular}{@{}p{50pt}|p{\columnwidth-50pt-2\tabcolsep}@{}}#2\end{tabular}}
\newcommand{\sectionNL}[0]{\\\multicolumn{2}{c}{}\vspace{1.5mm}\\}
\newcommand{\smallNL}[0]{\\\multicolumn{2}{c}{}\vspace{1mm}\\}
\newcommand{\entryNL}[0]{\\\vspace{1mm}\\}
% usage \workEntry{time}{position and workplace}{location}{extra detail}{content}
\newcommand{\workEntry}[5]{\raggedleft\textsc{#1}&\textbf{#2} #3\entryNL\raggedleft\footnotesize{#4}&\small{#5}}
% usage \projEntry{time}{title}{languages}{content}
\newcommand{\projEntry}[4]{\raggedleft\textsc{#1}&\textbf{#2} \textsc{#3}\entryNL&\small{#4}}
% usage \expEntry{time}{experience}
\newcommand{\expEntry}[2]{\raggedleft\textsc{#1}\vspace{-1mm}& \small #2}

%--------------------SECTIONS----------------------------
\resumeSec{Work Experience}{
\smallNL


\workEntry{2022}{Code Coach at theCoderSchool}{Irvine, CA}{}{
I am teaching students in elementary and middle school how to code in Python and Scratch. I learned how to explain programming concepts, in ways that anybody could understand.}
\sectionNL

\workEntry{2021}{Robotics Teacher for RoboQ}{Online}{}{
Remotely taught high school students arduino robotics. Each student learned how to program, solder, and cad. I bettered my communication skills by making sure each student understood the material.}
\sectionNL

\workEntry{2015-2018}{Tech Department at Green Polishing Solutions}{Canoga Park, CA}{}{
Set up, repaired, and maintained the company’s servers, computers, and printers- mostly involving installing software, removing viruses, and swapping parts. }

}

\resumeSec{Projects}{
\smallNL


\projEntry{2022}{Desktop Pet}{Python
{\href{https://github.com/BrandonComins/Desktop-Pet}{\color{links}{github.com/BrandonComins/Desktop-Pet}}}}{
Made an annoying desktop pet that will sometimes steal your mouse if you leave it idle. I re-learned how to use the tkinter library.
}
\sectionNL

\projEntry{2020}{Curdle Game Jam 5}{C/C++
{\href{https://github.com/BrandonComins/Curdle-Game-Jam-5}{\color{links}{github.com/BrandonComins/Game}}}}{
Entered a game jam. I Implemented a functioning camera and procedurally generated dungeon. I learned to use C\# and the debugger for the first time. 
}
\sectionNL

\projEntry{2020}{Drone}{C/C++
{\href{https://github.com/BrandonComins/Eng-7B-Drone}{\color{links}{github.com/BrandonComins/Drone}}}}{
Entered a class competition to construct a drone with sensors that determine when to deploy a servo-actuated payload. Despite initially struggling with the aerodynamics of the drone, I re-designed and re-built my drone four times until it could finally fly! I placed 3rd out of 50+ other groups. 
}
\sectionNL

\projEntry{2019}{Rocket Flight Computer}{C/C++
{\href{https://github.com/BrandonComins/Rocket}{\color{links}{github.com/BrandonComins/Rocket}}}}{
Constructed a flight computer that logged data and used sensors to determine when to deploy the parachute. Although this was an annual project for my high school, I was the first person in the school's history to take the challenge of making a flight computer instead of buying one.
This project taught me problem solving through debugging, wire management and soldering, and about mosfets.
 }
\sectionNL

\projEntry{2019}{Fruit Piano}{C/C++
{\href{https://github.com/BrandonComins/FruitPiano}{\color{links}{github.com/BrandonComins/FruitPiano}}}}{
Went to a science fair at Darby Elementary School to demonstrate STEM and robotics. I made a fruit-actuated "piano," to show that wiring and electronics can be fun. I learned about using unconventional methods, such as fruit, to close a wiring loop.
}
\sectionNL

\projEntry{2019}{RC Car}{C/C++
{\href{https://github.com/BrandonComins/RC-Car}{\color{links}{github.com/BrandonComins/RC-Car}}}}{
As a camp counselor, I constructed a remote controlled car using an Arduino. I made this project because I wanted to show the kids that robotics isn't just work, but is also fun.  This project also taught me how to use speed controllers and a bluetooth module for the first time. 
}


%\projEntry{2015-2019}{FRC Robotics}{Java
%{\href{https://github.com/team4element}{\color{links}{github.com/team4element}}}{
%As a member of First Robotics Competition, I took on programming the robot. Here I actuated motors, via xbox controller in teleoperation, as well as controlled the robot autonomously. In autonomous mode, the robot could vision track, and path find.
%}

}


\resumeSec{Technical Skills and Qualifications}{
\smallNL
\raggedleft Languages& \textsc{Java, Python, C++, C, Verilog, VHDL, MIPS Assembly, Regex, Embeded Systems, Latex}
\smallNL
\raggedleft Software& Xilinx Vivado, Git, Visual Studio, MATLAB, Solidworks, Simplify3D, Arduino, ROS2
}

\begin{tabular}{@{}p{.65\textwidth}p{.35\textwidth-2\tabcolsep}@{}}
\resumeSec{Education}{
\sectionNL
\workEntry{2019-2023}{University of California, Irvine}{Irvine, CA}{}{
Graduation in 2023, \textsc{Computer Science Engineering B.S}}
\smallNL
\workEntry{2019}{Los Angeles Pierce College}{Woodland Hills, CA}{}{
Concurrent enrollment with high school (HTLA)}}
&
\section{Relevant Coursework}
\vspace{1.7mm}\raggedright\small
% Electrical,
Electrical Devices and Systems,
Organization of Digital Computers,
%Programming
Python Programming,
Advanced C,
Data Structures,
Intro to computer graphics,
Discrete Mathematics \& Probability Theory,
Linear Algebra \& Differential Equations,
Calculus 3,
% ETC
Intro to Software Engineering
\end{tabular}

\resumeSec{Additional Experience}{
\smallNL


%\expEntry{2015-2018}{Tech Department at Green Polishing Solutions}
%\smallNL
\expEntry{2022}{Software Lead for Legacy Robotics, UCI's Robomaster University  League's Team}
\smallNL
\expEntry{2020-2022}{Mentor High Tech Los Angeles in FRC (First Robotics Competition)}
\smallNL
\expEntry{2019}{Computer Teacher at One Generation}
\smallNL
\expEntry{2019}{Team Captain of High School FRC team (First Robotics Competition)}
\smallNL
\expEntry{2019}{Mentor Darby Elementary in Lego Robotics (First Lego League)}
\smallNL
\expEntry{2018}{Camp Counselor for Lego Robotics Camp}
\smallNL
\expEntry{2016-2018}{Volunteer at Motion Picture Funding Hospital}
}
\end{document}